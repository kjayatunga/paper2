
\section{Limit where equation 8 predicts a frequency of zero}

Equation 8 states that

\begin{equation}
  \label{eqn:eigs}
  \lambda_{1,2}= -\frac{1}{2}\frac{c-\frac{1}{2}\rho U\mathcal{A}a_1}{(m+m_a)}\pm\frac{1}{2}\sqrt{\left[\frac{c-\frac{1}{2}\rho U\mathcal{A}a_1}{(m+m_a)}\right]^2-4\frac{k}{(m+m_a)}}.
\end{equation}

So, a frequency can be estimated from the square root term, namely

\begin{equation}
  \label{eqn:freq}
  f = \sqrt{\left[\frac{c-\frac{1}{2}\rho U\mathcal{A}a_1}{(m+m_a)}\right]^2-4\frac{k}{(m+m_a)}}.
\end{equation}

We want to know the limit where this frequency goes to zero in terms
of our non-dimensional parameters \massstiff and \massdamp. So, we can
non-dimensionalize the above equation as follows:

\begin{align}
  \cstar&=\frac{cD}{\mtot U} \\
  \mstar&=\frac{\mtot}{\rho \mathcal{A}D} \\
  \ustar&=\frac{U}{f_N D} = \frac{2\pi U}{\sqrt{k/\mtot}D}
\end{align}
where
\begin{align}
  \mathcal{A} &= DL \\
  \mtot & = m + m_a
\end{align}
where $m$ is the mass of the body and $m_a$ is the added mass.

So, these nondimensional parameters can be reorganized to show that
\begin{align}
  \frac{k}{\mtot} &= \left(\frac{U}{D}\right)^2\left(\frac{2\pi}{U*}\right)^2 \\
  \frac{c}{\mtot} &= c*\frac{U}{D}
\end{align}

These can then be substituted into equation (\ref{eqn:freq}) to give
\begin{equation}
f = \sqrt{\left[c*\left(\frac{U}{D}\right) - \frac{1}{2}\frac{a_1}{m*}\left(\frac{U}{D}\right)\right]^2 - 4\left(\frac{U}{D}\right)^2\frac{2\pi}{U*}}.
\end{equation}
This can then be rewritten as
\begin{equation}
  f = \sqrt{\left(\frac{U}{D}\right)^2\left(c*-\frac{a_1}{2m*}\right)^2 - 4\left(\frac{U}{D}\right)^2\left(\frac{2\pi}{U*}\right)^2}.
\end{equation}
You can then take the factor of $U/D$ to the left-hand side to get
\begin{equation}
  \frac{fD}{U} = \sqrt{\left(c*-\frac{a_1}{2m*}\right)^2 - 4\left(\frac{2\pi}{U*}\right)^2}.
\end{equation}
Expanding terms gives
\begin{equation}
  \frac{fD}{U} = \sqrt{c*^2 - \frac{2c*a_1}{2m*} + \frac{a_1^2}{4m*^2} - \frac{16\pi^2}{U*^2}}.
\end{equation}
Multiplying through by $m*^2$ gives
\begin{equation}
  \frac{fD}{U} = \sqrt{c*^2m*^2 - c*m*a_1 + \frac{a_1^2}{4} - \frac{16\pi^2m*^2}{U*^2}}.
\end{equation}

Now, we have defined
\begin{align}
  \Pi_1 &= \frac{4\pi^2m*^2}{U*^2} \\
  \Pi_2 &= m*c*,
\end{align}
so we can substitute these in to arrive at the final statement of
\begin{equation}
  \label{eqn_freq_final}
  \frac{fD}{U} = \sqrt{\Pi_2^2 - \Pi_2a_1 + \frac{a_1^2}{4} - 4\Pi_1}.
\end{equation}

So, by setting $f=0$, the relationship between $\Pi_1$ and $\Pi_2$ at
the limit can be found.

\section{Finding the terminal velocity of the body when no frequency
  is predicted by equation 8}

For very small $Pi_1$ where no frequency is predicted by equation 8,
we can assume that the body quickly accelerates to a velocity where
the lift force is blanced by the damping force. While the displacement
is small, the spring force is basically negligible. Also, for the
velocity to saturate (reach a constant value), we need only one
nonlinear term in the equation, and so we retain only up to the cubic
velocity term in the lift force to give
\begin{equation}
  c\dot{y} = \frac{1}{2}\rho U^2\mathcal{A}\left[a_1\left(\frac{\dot{y}}{U}\right) + a_3\left(\frac{\dot{y}}{U}\right)^3\right].
\end{equation}
Rearranging and dividing by $\dot{y}$ gives
\begin{equation}
  \left(\frac{1}{2}\rho U\mathcal{A}a_1 - c\right) + \frac{1}{2}\rho\frac{1}{U}\mathcal{A}a_3\dot{y}^2 = 0,
\end{equation}
So that the terminal velocity $\dot{y}$ is given by
\begin{equation}
  \dot{y} = \pm\sqrt{-\frac{(1/2)\rho U^2\mathcal{A}a_1 - cU}{(1/2)\rho\mathcal{A}a_3}}.
\end{equation}
This can be written as
\begin{equation}
  \dot{y} = \pm\sqrt{U^2\frac{a_1}{a_3} - U^2\frac{c}{(1/2)\rho U\mathcal{A}a_3}}
\end{equation}
or
\begin{equation}
  \frac{\dot{y}}{U} = \pm\sqrt{\frac{a_1}{a_3} - \frac{c}{(1/2)\rho U\mathcal{A}a_3}}.
\end{equation}

Finally, the definition of $\Pi_2$ can be substituted into the last term to give
\begin{equation}
  \frac{\dot{y}}{U} = \pm\sqrt{\frac{1}{a_3}(a_1 - 2\Pi_2)},
\end{equation}
hence the terminal, or maximum velocity is a function of $\Pi_2$ only.

